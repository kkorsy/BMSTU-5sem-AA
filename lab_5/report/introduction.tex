\chapter*{ВВЕДЕНИЕ}
\addcontentsline{toc}{chapter}{ВВЕДЕНИЕ}

\textbf{Целью} данной работы является изучение асинхронного взаимодействия потоков вычисления на примере конвейерных вычислений. Для достижения поставленной цели необходимо выполнить следующие задачи:
\begin{itemize}[label=---]
	\item описать конвейерную обработку;
	\item описать алгоритмы Кнута~---~Морриса~---~Пратта и Бойера~---~Мура;
	\item реализовать конвейерную обработку с указанными алгоритмами;
	
	\item провести тестирование по методу черного ящика для реализации конвейера;
	
	\item провести сравнительный анализ зависимости времени обработки от количества заявок;
	
	\item описать и обосновать полученные результаты в отчете о выполненной лабораторной работе, выполненного как расчетно-пояснительная записка к работе.
\end{itemize}