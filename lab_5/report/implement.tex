\chapter{Технологическая часть}

В данном разделе будут рассмотрены средства реализации, а также представлены листинги реализаций алгоритмов Кнута~---~Морриса~---~Пратта и Бойера~---~Мура, а также конвейерной обработки.

\section{Средства реализации}

В данной работе для реализации был выбран язык программирования \textit{C++} \cite{cpp}. В текущей лабораторной работе требуется замерить процессорное
время для выполняемой программы, а также реализовать принципы многопоточного алгоритма. Все эти инструменты присутствуют в выбранном языке программирования.
Время замерено с помощью функции $std::chrono::system\_clock::now()$
из библиотеки \textit{chrono} \cite{chrono}. 

\section{Описание используемых типов данных}

При реализации будут использованы следующие типы и структуры данных:
\begin{itemize}[label=---]
	\item \textit{std::set} --- для результирующего набора найденных индексов;
	\item \textit{std::string} --- для исходной строки и искомой подстроки;
	\item \textit{request\_t} --- структура заявки, содержащая имя файла для поиска, искомую подстроку, флаг завершения;
	\item \textit{std::queue<request\_t>} --- для очередей заявок;
	\item \textit{std::vector<std::thread>} --- для потоков.
\end{itemize}

\section{Сведения о модулях программы}

Программа состоит из двух модулей:
\begin{itemize}[label={---}]
	\item \textit{main.cpp} --- файл, содержащий весь служебный код;
	\item \textit{conveyor.cpp} --- файл, содержащий конвейерную обработку.
\end{itemize}

\section{Реализация алгоритмов}

В листингах~\ref{lst:kmp1}~--~\ref{lst:conveyor2} представлены реализации алгоритмов Кнута~---~Морриса~---~Пратта и Бойера~---~Мура, а также конвейерной обработки.

\lstinputlisting[label=lst:kmp1, caption={Реализация алгоритма Кнута~---~Морриса~---~Пратта (начало)}, firstline=199, lastline=227]{../conveyor.cpp}

\lstinputlisting[label=lst:kmp2, caption={Реализация алгоритма Кнута~---~Морриса~---~Пратта (окончание)}, firstline=227, lastline=247]{../conveyor.cpp}

\lstinputlisting[label=lst:bm1, caption={Реализация алгоритма Бойера~---~Мура (начало)}, firstline=248, lastline=262]{../conveyor.cpp}

\lstinputlisting[label=lst:bm2, caption={Реализация алгоритма Бойера~---~Мура (продолжение)}, firstline=262, lastline=301]{../conveyor.cpp}

\lstinputlisting[label=lst:bm3, caption={Реализация алгоритма Бойера~---~Мура (окончание)}, firstline=301, lastline=312]{../conveyor.cpp}

\lstinputlisting[label=lst:conveyor1, caption={Реализация алгоритма конвейерной обработки (начало)}, firstline=6, lastline=30]{../conveyor.cpp}

\lstinputlisting[label=lst:conveyor2, caption={Реализация алгоритма конвейерной обработки (окончание)}, firstline=30, lastline=44]{../conveyor.cpp}

\section{Вывод}

Были представлены реализации алгоритмов Кнута~---~Морриса~---~Пратта и Бойера~---~Мура, а также конвейерной обработки, которые были описаны в предыдущем разделе. Также в данном разделе была приведена информация о выбранных средствах для разработки алгоритмов.