\chapter*{Введение}
\addcontentsline{toc}{chapter}{Введение}

В математике и программировании часто приходится прибегать к использованию матриц. Существует огромное количество областей их применения в этих сферах. Например, матрицы активно используются при выводе различных формул в физике:

\begin{itemize}[label=---]
	\item градиент;
	\item дивергенция;
	\item ротор.
\end{itemize}

Также часто применяются и операции над матрицами --- сложение, возведение в степень, умножение. При различных задачах размеры матрицы могут достигать больших значений. Поэтому оптимизация операций работы над матрицами является важной задачей в программировании. Об оптимизации операции умножения пойдет речь в данной лабораторной работе.

\textbf{Целью} данной работы является описание, реализация и исследование алгоритмов умножения матриц --- классический алгоритм, алгоритм Винограда и оптимизированный алгоритм Винограда. Для достижения поставленной цели необходимо выполнить следующие задачи:

\begin{itemize}[label=---]
	\item описать классический алгоритм умножения матриц, алгоритмы Винограда и его оптимизации;
	
	\item реализовать классический алгоритм умножения матриц, алгоритмы Винограда и его оптимизации;
	
	\item провести тестирование по методу черного ящика для реализаций классического алгоритма умножения матриц, алгоритмов Винограда и его оптимизации;
	
	\item провести сравнительный анализ по времени и по памяти реализаций классического алгоритма умножения матриц, алгоритмов Винограда и его оптимизации;
	
	\item описать и обосновать полученные результаты в отчете о выполненной лабораторной работе, выполненного как расчетно-пояснительная записка к работе.
\end{itemize}