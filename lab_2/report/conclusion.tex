\chapter*{Заключение}
\addcontentsline{toc}{chapter}{Заключение}

В результате исследования было определено, что при четных размерах матриц реализация классического алгоритма умножения матриц работает быстрее алгоритма Винограда при размерах матриц больше $[50\times 50]$. При нечетных размерах --- при размерах больше $[21 \times 21]$. 
Также было получено, что оптимизированный алгоритм Винограда требует больше памяти, чем классический: затраты на $2\cdot sizeof(int)$ байт больше. Но наименьшие затраты по памяти у классического алгоритма умножения матриц. 

Цель, которая была поставлена в начале лабораторной работы была достигнута: описаны, реализованы и исследованы алгоритмы умножения матриц: классический, Винограда и его оптимизации. В ходе выполнения были решены все задачи:

\begin{itemize}[label=---]
	\item описаны алгоритмы Винограда и его оптимизации;
	
	\item реализованы алгоритмы Винограда и его оптимизации;
	
	\item проведено тестирование по методу черного ящика для реализаций алгоритмов Винограда и его оптимизации;
	
	\item проведен сравнительный анализ по времени и по памяти реализаций алгоритмов Винограда и его оптимизации;
	
	\item описаны и обоснованы полученные результаты в отчете о выполненной лабораторной работе, выполненного как расчетно-пояснительная записка к работе.
\end{itemize}