\chapter{Аналитическая часть}
В данном разделе будут рассмотрены алгоритмы умножения матриц: классический, Винограда и его оптимизации.

\section{Классический алгоритм умножения матриц}

Произведением матриц $A$ и $B$ называется матрица $C$ такая, что число строк и столбцов матрицы $C$ равно количеству строк матрицы $A$ и столбцов матрицы $B$ соответственно.

Необходимым условием умножения двух матриц является равенство количества столбцов первой матрицы количеству строк второй матрицы.

Пусть даны матрицы $A$ [$a\times b$] и $B$ [$b\times d$]:
\begin{equation}
	A = \left(
	\begin{array}{cccc}
		a_{11} & a_{12} & \ldots & a_{1n}\\
		a_{21} & a_{22} & \ldots & a_{2n}\\
		\vdots & \vdots & \ddots & \vdots\\
		a_{n1} & a_{n2} & \ldots & a_{nn}
	\end{array}
	\right),
\end{equation}

\begin{equation}
	B = \left(
	\begin{array}{cccc}
		b_{11} & b_{12} & \ldots & b_{1n}\\
		b_{21} & b_{22} & \ldots & b_{2n}\\
		\vdots & \vdots & \ddots & \vdots\\
		b_{n1} & b_{n2} & \ldots & b_{nn}
	\end{array}
	\right),
\end{equation}

тогда произведением матриц $A$ и $B$ будет считаться матрица $C$ [$a$ x $d$]:

\begin{equation}
	C = \left(
	\begin{array}{cccc}
		c_{11} & c_{12} & \ldots & c_{1n}\\
		c_{21} & c_{22} & \ldots & c_{2n}\\
		\vdots & \vdots & \ddots & \vdots\\
		c_{n1} & c_{n2} & \ldots & c_{nn}
	\end{array}
	\right),
\end{equation}

где

\begin{equation}
	\label{eq:classic}
	C_{ij} = \sum\limits_{k=1}^m a_{ik}b_{kj},\ i = 1, 2, ..., a;\ j = 1, 2, ..., d. 
\end{equation}

Классический алгоритм \cite{classic} умножения двух матриц работает по формуле~(\ref{eq:classic}).

\section{Алгоритм Винограда}

Пусть умножаются матрицы~$A$, содержащая $M$~строк и $Q$~столбцов, и $B$, содержащая $Q$~строк и $N$~столбцов. Результатом умножения будет матрица~$C$ с $M$~стоками и $N$~столбцами.

Каждый элемент $C_{ij}$ матрицы~$C$ вычисляется как произведение строки~$i$ первой матрицы на столбец~$j$ второй.

\begin{multline}
C_{ij} = 
\left[ {\begin{array}{cccc}
		a_{i1} & a_{i2} & a_{i3} & a_{i4}
\end{array}}
\right] 
\cdot
\left[ {\begin{array}{c}
		b_{1j}\\
		b_{2j}\\
		b_{3j}\\
		b_{4j}\\
\end{array}}
\right] = 
a_{i1}b_{1j} + a_{i2}b_{2j} +a_{i3}b_{3j} + a_{i4}b_{4j}
\end{multline}

Результат умножения представим в виде:

\begin{equation}
\begin{split}
	&a_{i1}b_{1j} + a_{i2}b_{2j} +a_{i3}b_{3j} + a_{i4}b_{4j} =\\
	&(a_{i1} + b_{2j})(a_{i2} + b_{1j}) + (a_{i3} + b_{4j})(a_{i4} + b_{3j}) -\\
	&- a_{i1}a_{i2} - a_{i3}a_{i4} - b_{1j}b_{2j} - b_{3j}b_{4j}
\end{split}
\end{equation}

При этом слагаемые $$- a_{i1}a_{i2} - a_{i3}a_{i4} - b_{1j}b_{2j} - b_{3j}b_{4j}$$ могут быть вычислены заранее и использованы повторно.

Таким образом, алгоритм Винограда \cite{vinorgad} снижает долю операций умножения в произведении матриц.

\section{Оптимизированный алгоритм Винограда}

Оптимизация будет включать в себя:
\begin{itemize}[label=---]
	\item замену операций вида $x = x + k$ на $x += k$;
	\item замену умножения на 2 на побитовый сдвиг влево;
	\item предвычисление некоторых слагаемых для алгоритма.
\end{itemize}

\section{Вывод}

В данном разделе было дано математическое описание классического алгоритма умножения матриц, алгоритма Винограда и описана суть его оптимизации.

К разрабатываемой программе предъявляются следующие требования.

\begin{enumerate}
	\item Программа должна предоставлять функциональность раcчета произведения матриц классическим алгоритмом, алгоритмом Винограда и его оптимизации.
	
	\item Реализуемое ПО будет работать в двух режимах --- пользовательском, в котором можно выбрать алгоритм и вывести для него рассчитанное значение, а также экспериментальном режиме, в котором можно произвести сравнение реализаций алгоритмов по времени работы на различных входных данных.
	
	\item В первом режиме в качестве входных данных в программу будет подаваться две матрицы, также реализовано меню для вызова алгоритмов и замеров времени. Программа должна корректно обрабатывать случай ввода матриц с размерами, для которых умножение невозможно.
	
	\item Во втором режиме будет происходить измерение процессорного времени работы программы, будут построены зависимости времени работы от совпадающих размеров квадратных матриц. Входные матрицы будут сгенерированы автоматически для заданного совпадающего размера матрицы.
\end{enumerate}