\chapter{Аналитическая часть}
В данном разделе будут рассмотрены задача коммивояжера и используемые для ее решения алгоритмы: алгоритм полного перебора и муравьиный алгоритм. 

\section{Задача коммивояжера}

Коммивояжер (фр. commis voyageur) --- бродячий торговец.  Коммивояжеру, чтобы распродать товары, следует объехать $n$ пунктов и в конце концов вернуться в исходный пункт. Суть задачи коммивояжера --- определить наиболее выгодный маршрут объезда \cite{kom-task}. В качестве меры выгодности маршрута может служить суммарное время в пути, суммарная стоимость дороги, или, в простейшем случае, длина
маршрута.
В описываемой задаче рассматривается несколько городов и матрица затрачиваемого времени на дорогу между ними, причем суммарное время пути не должно превышать 80 дней в человеческом измерении.

\section{Алгоритм полного перебора}

Алгоритм полного перебора для решения задачи коммивояжера предполагает рассмотрение всех возможных путей в графе и выбор наименьшего из них \cite{kom-task-all}.

Преимуществом данного алгоритма является гарантия нахождения глобального оптимума. Недостатком --- трудоемкость $O(n!)$. 

\section{Муравьиный алгоритм}

Муравьиный алгоритм  --- метод решения задачи коммивояжера, в основе которого лежит моделирование поведения колонии муравьев \cite{ant-alg}.
Каждый муравей определяет для себя маршрут, который необходимо пройти на основе феромона, который он ощущает во время прохождения, каждый муравей оставляет феромон на своем пути, чтобы остальные муравьи могли по нему ориентироваться. В результате при прохождении каждым муравьем различного маршрута наибольшее число феромона остается на оптимальном пути.

Пусть муравей имеет следующие характеристики:
\begin{itemize}[label=---]
	\item зрение --- способен определить длину ребра;
	\item память --- запоминает пройденный маршрут;
	\item обоняние --- чувствует феромон.
\end{itemize}

Также введем целевую функцию \eqref{d_func} и формулу вычисления вероятности перехода в заданную точку \eqref{posib}.
\begin{equation}
	\label{d_func}
	\eta_{ij} = 1 / D_{ij},
\end{equation}
где $D_{ij}$ — время, затрачиваемое на дорогу из текущего пункта $i$ до заданного пункта $j$.

\begin{equation}
	\label{posib}
	P_{kij} = \begin{cases}
		\frac{\tau_{ij}^a\eta_{ij}^b}{\sum_{q=1}^m \tau^a_{iq}\eta^b_{iq}}, \textrm{вершина не была посещена ранее муравьем k,} \\
		0, \textrm{иначе}
	\end{cases}
\end{equation}
где $a$ --- коэффициент жадности решения, $b$ --- коэффициент стадности решения, $\tau_{ij}$ --- расстояния от города $i$ до $j$, $\eta_{ij}$ --- количество феромонов на ребре $ij$.

После завершения движения всех муравьев, формула обновляется феромон по формуле \eqref{update_phero_1}:
\begin{equation}
	\label{update_phero_1}
	\tau_{ij}(t+1) = (1-\rho)\tau_{ij}(t) + \Delta \tau_{ij}.
\end{equation}

При этом
\begin{equation}
	\label{update_phero_2}
	\Delta \tau_{ij} = \sum_{k=1}^N \tau^k_{ij},
\end{equation}
где
\begin{equation}
	\label{update_phero_3}
	\Delta\tau^k_{ij} = \begin{cases}
		Q/L_{k}, \textrm{ребро посещено k-ым муравьем,} \\
		0, \textrm{иначе}
	\end{cases}
\end{equation}

Если феромон на дуге обнулится, то обнулится и вероятность перехода в соответствующий город, что недопустимо. Поэтому в конце модификации феромона требуется выполнить дополнительную проверку: если феромон на дуге упал ниже малой константы, то его требуется откатить до этой константы.


Преимуществом данного алгоритма является трудоемкость:
$$O(t_{max}* \max(m, n))$$
где $t_{max}$ --- время жизни колонии, $m$ --- количество муравьев, $n$ -- размер графа.
Недостатком --- отсутствие гарантии нахождения глобального оптимума. 

\section{Вывод}

В данном разделе были теоретически разобраны алгоритмы полного перебора и муравьиного решения задачи коммивояжера.

В таблице \ref{tbl:cmp} представлены результаты сравнения указанных алгоритмов.
\begin{center}
\captionsetup{justification=raggedright,singlelinecheck=off}
\begin{longtable}{|c|c|c|}
	\caption{\label{tbl:cmp}Сравнение алгоритмов}\\
	\hline
	Критерий & \begin{tabular}{c}
		Алгоритм\\
		полного перебора
	\end{tabular} & Муравьиный алгоритм\\
	\hline
	\begin{tabular}{c}
		Гарантия нахождения\\
		глобального оптимума
	\end{tabular}
	  & Да & Нет\\ 
	\hline
	Трудоемкость & $O(n!)$ & $O(t_{max}* \max(m, n))$\\
	\hline
\end{longtable}
\end{center}

К разрабатываемой программе предъявляются следующие требования:
\begin{enumerate}
	\item Программа должна предоставлять функциональность рассчета минимального времени пути, а также самого пути;
	
	\item Реализуемое ПО будет работать в двух режимах --- пользовательском, в котором можно выбрать алгоритм и вывести для него результат, а также экспериментальном режиме, в котором можно произвести сравнение реализаций алгоритмов по времени работы на различных входных данных;
	
	\item В первом режиме в качестве входных данных в программу будет подаваться файл с матрицей времен, затрачиваемых на путь из одного города в другой, также реализовано меню для вызова алгоритмов и замеров времени. Программа должна корректно обрабатывать случай ввода несуществующего имени файла;
	
	\item Во втором режиме будет происходить измерение процессорного времени работы программы, будут построены зависимости времени работы от размера матрицы. Матрица будет сгенерирована автоматически для заданного размера матрицы.
\end{enumerate}