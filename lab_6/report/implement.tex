\chapter{Технологическая часть}

В данном разделе будут рассмотрены средства реализации, а также представлены листинги реализаций алгоритмов полного перебора и муравьиного.

\section{Средства реализации}

В данной работе для реализации был выбран язык программирования $Python$~\cite{python-lang}. В текущей лабораторной работе требуется замерить процессорное время для выполняемой программы, а также построить графики. Все эти инструменты присутствуют в выбранном языке программирования.

Время работы было замерено с помощью функции \textit{process\_time} из библиотеки $time$~\cite{python-lang-time}.

\section{Описание используемых типов данных}

При реализации будут использованы следующие типы и структуры данных:
\begin{itemize}[label=---]
	\item размер матрицы смежности --- целое число типа \textit{int};
	\item название файла --- строка типа \textit{str};
	\item коэффициенты $\alpha$, $\beta$, \textit{evaporation\_koef} --- числа типа \textit{float};
	\item матрица времен, затрачиваемых на дорогу из одного города в другой --- матрица типа \textit{int}.
\end{itemize}

\section{Сведения о модулях программы}

Программа состоит из двух модулей:
\begin{itemize}[label={---}]
	\item \textit{main.py} --- файл, содержащий весь служебный код;
	\item \textit{algorythms.py} --- файл, содержащий реализации алгоритмов.
\end{itemize}

\section{Реализация алгоритмов}

В листингах~\ref{lst:full}~---~\ref{lst:ant} представлены реализации алгоритмов полного перебора и муравьиного.

\lstinputlisting[label=lst:full, caption={Реализация алгоритма полного перебора}, firstline=8, lastline=31]{../algorythms.py}

\lstinputlisting[label=lst:ant, caption={Реализация муравьиного алгоритма}, firstline=129, lastline=160]{../algorythms.py}

\section{Функциональные тесты}

В таблице \ref{tbl:functional_test} приведены тесты для функций, реализующих алгоритмы полного перебора и муравьиного. В качестве результата ожидается число и массив: суммарное время пути и маршрут. Тесты \textit{для всех алгоритмов} пройдены успешно. 

\newpage
\begin{center}
	\captionsetup{justification=raggedright,singlelinecheck=off}
	\begin{longtable}[c]{|c|c|c|c|c|}
		\caption{Функциональные тесты\label{tbl:functional_test}} \\ \hline
		Матрица & Ожидаемый результат & Результат программы \\
		\hline
		$ \begin{pmatrix}
			0 &  4 &  2 &  1 & 7 \\
			4 &  0 &  3 &  7 & 2 \\
			2 &  3 &  0 & 10 & 3 \\
			1 &  7 & 10 &  0 & 9 \\
			7 &  2 &  3 &  9 & 0
		\end{pmatrix}$ &
		15, [0, 2, 4, 1, 3, 0] &
		15, [0, 2, 4, 1, 3, 0] \\
		
		$ \begin{pmatrix}
			0 & 1 & 2 \\
			1 & 0 & 1 \\
			2 & 1 & 0	
		\end{pmatrix}$ &
		4, [0, 1, 2, 0] &
		4, [0, 1, 2, 0] \\
		
		$ \begin{pmatrix}
			0 & 15 & 19 & 20 \\
			15 &  0 & 12 & 13 \\
			19 & 12 &  0 & 17 \\
			20 & 13 & 17 &  0
		\end{pmatrix}$ &
		64, [0, 1, 2, 3, 0] &
		64, [0, 1, 2, 3, 0] \\
		\hline
	\end{longtable}
\end{center}

\section{Вывод}

Были представлены реализации алгоритмов полного перебора и муравьиного, которые были описаны в предыдущем разделе. Также в данном разделе была приведена информация о выбранных средствах для разработки алгоритмов.