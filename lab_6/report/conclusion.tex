\chapter*{Заключение}
\addcontentsline{toc}{chapter}{Заключение}

В результате было получено, что реализация муравьиного алгоритма работает быстрее реализации алгоритма полного перебора при размерах матрицы больше 8. А оптимальными параметрами для муравьиного алгоритма являются $\alpha = 0.2, \rho = 0.4, количество дней = 500$.

Цель, которая была поставлена в начале лабораторной работы была достигнута: исследованы алгоритмы полного перебора и муравьиного. В ходе выполнения были решены все задачи:

\begin{itemize}[label=---]
	\item описаны алгоритм полного перебора и муравьиный алгоритм;
	
	\item реализованы указанные алгоритмы;
	
	\item проведено тестирование по методу черного ящика для реализаций указанных алгоритмов;
	
	\item проведен сравнительный анализ по времени и по памяти реализаций указанных алгоритмов;
	
	\item описаны и обоснованы полученные результаты в отчете о выполненной лабораторной работе, выполненного как расчетно-пояснительная записка к работе.
\end{itemize}