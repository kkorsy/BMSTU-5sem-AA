\chapter{Конструкторская часть}

В этом разделе будут представлено описание используемых типов данных,
а также схемы алгоритмов вычисления расстояний Левенштейна и Дамерау~--- Левенштейна.

\section{Сведения о модулях программы}

Программа состоит из четырех модулей:
\begin{itemize}[label={---}]
	\item \textit{main.py} --- файл, содержащий весь служебный код;
	\item \textit{algorithms.py} --- файл, содержащий реализации алгоритмов;
	\item \textit{graph.py} --- файл, содержащий код для построения графиков; 
	\item \textit{tests.py} --- файл, содержащий модульные тесты.
\end{itemize}

\section{Разработка алгоритмов}

На рисунках~\ref{img:DLrec} ---~\ref{img:Opt2} представлены схемы алгоритмов вычисления расстояний Левенштейна и Дамерау --- Левенштейна.

\inputPdf{DLrec}{Схема рекурсивного алгоритма нахождения расстояния Дамерау --- Левенштейна}

\inputPdf{DLcache1}{Схема рекурсивного алгоритма нахождения расстояния Дамерау --- Левенштейна с использованием кэша в виде матрицы (часть 1)}

\inputPdf{DLcache2}{Схема рекурсивного алгоритма нахождения расстояния Дамерау --- Левенштейна с использованием кэша в виде матрицы (часть 2)}

Матричные алгоритмы нахождения расстояний Левенштейна и Дамерау --- Левенштейна можно описать в виде одной схемы алгоритмов, представленной на рисунках~\ref{img:FullM1}--\ref{img:FullM2}. Если \textit{flag == true}, то будет вычислено расстояние Дамерау --- Левенштейна, иначе --- расстояние Левенштейна.

\inputPdf{FullM1}{Схема матричного алгоритма нахождения расстояний Левенштейна и Дамерау --- Левенштейна (часть 1)}

\inputPdf{FullM2}{Схема матричного алгоритма нахождения расстояний Левенштейна и Дамерау --- Левенштейна (часть 2)}

Аналогично для оптимизированного матричного алгоритма нахождения расстояний Левенштейна и Дамерау --- Левенштейна. Полученная схема представлена на рисунках~\ref{img:Opt1}--\ref{img:Opt2}.

\inputPdf{Opt1}{Схема оптимизированного матричного алгоритма нахождения расстояний Левенштейна и Дамерау --- Левенштейна (часть 1)}

\inputPdf{Opt2}{Схема оптимизированного матричного алгоритма нахождения расстояний Левенштейна и Дамерау --- Левенштейна (часть 2)}

\section{Классы эквивалентности тестирования}

Для тестирования выделены следующие классы эквивалентности:
\begin{enumerate}
	\item Ввод двух пустых строк;
	\item Одна из строк пустая;
	\item Ввод одинаковых строк;
	\item Расстояния Левенштейна и Дамерау --- Левенштейна равны;
	\item Расстояния Левенштейна и Дамерау --- Левенштейна не равны.
\end{enumerate}

\section{Использование памяти}

Пусть n --- длина строки $S_1$, m --- длина строки $S_2$.

\subsection{Рекурсивный алгоритм}

Для каждого вызова:
\begin{itemize}[label={---}]
	\item Расстояние Левенштейна:
	\begin{itemize}
		\item для $S_1$, $S_2$: (n + m) * sizeof(char);
		\item доп. переменные: 1 * sizeof(int);
		\item адрес возврата.
	\end{itemize}
	
	\item Затраты по памяти для нахождения расстояния Дамерау --- Левенштейна равны затратам для расстояния Левенштейна.
\end{itemize}

Высота дерева вызовов: n + m.

\subsection{Рекурсивный алгоритм с использованием кэша в виде матрицы}

\begin{itemize}[label={---}]
	\item Расстояние Левенштейна:
	\begin{itemize}[label=---]
		\item для матрицы: ((n + 1) * (m + 1)) * sizeof(int);
		\item Для каждого вызова:
		\begin{itemize}[label=---]
			\item для $S_1$, $S_2$: (n + m) * sizeof(char);
			\item для n, m: 2 * sizeof(int);
			\item доп. переменные: 1 * sizeof(int);
			\item ссылка на матрицу: 8 байт;
			\item адрес возврата.
		\end{itemize}
	\end{itemize}
	
	\item Затраты по памяти для нахождения расстояния Дамерау --- Левенштейна равны затратам для расстояния Левенштейна.
\end{itemize}

\subsection{Матричный алгоритм}

\begin{itemize}[label={---}]
	\item Расстояние Левенштейна:
	\begin{itemize}[label=---]
		\item для матрицы: ((n + 1) * (m + 1)) * sizeof(int);
		\item для S1, S2: (n + m) * sizeof(char);
		\item для n, m: 2 * sizeof(int);
		\item адрес возврата.
	\end{itemize}
	
	\item Затраты по памяти для нахождения расстояния Дамерау --- Левенштейна равны затратам для расстояния Левенштейна.
\end{itemize}

\subsection{Оптимизированный матричный алгоритм}

\begin{itemize}[label={---}]
	\item Расстояние Левенштейна:
	\begin{itemize}
		\item для $S_1$, $S_2$: (n + m) * sizeof(char);
		\item для n, m: 2 * sizeof(int);
		\item 2 строки матрицы: 2 * (n + 1) * sizeof(char);
		\item адрес возврата.
	\end{itemize}
	
	\item Расстояние Дамерау --- Левенштейна:
	\begin{itemize}
		\item для $S_1$, $S_2$: (n + m) * sizeof(char);
		\item для n, m: 2 * sizeof(int);
		\item 3 строки матрицы: 3 * (n + 1) * sizeof(char);
		\item адрес возврата.
	\end{itemize}
\end{itemize}

\section*{Вывод}
Исходя из сравнения, представленного выше, меньше всего памяти требуется для оптимизированного матричного алгоритма, так как для него необходимы только 2 строки для вычисления расстояния Левенштейна и 3 строки для вычисления расстояния Дамерау --- Левенштейна.

\section*{Вывод}

В данном разделе были представлено описание используемых типов данных, а также схемы алгоритмов, рассматриваемых в лабораторной работе.



