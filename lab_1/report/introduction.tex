\chapter*{Введение}
\addcontentsline{toc}{chapter}{Введение}

Существует несколько важных задач, для решения которых нужны алгоритмы сравнения строк. Об этих алгоритмах и пойдет речь в данной работе. 
Подобные алгоритмы используются при \cite{dlbook}:

\begin{itemize}[label={---}]
	\item исправлении ошибок в тексте, предлагая заменить введенное слово с ошибкой на наиболее подходящее;
	
	\item поиске слова в тексте по подстроке;
	
	\item сравнении целых текстовых файлов. \newline
\end{itemize}

\textbf{Целью данной работы} исследование алгоритмов нахождения расстояний Левенштейна и Дамерау --- Левенштейна. 
Для достижения поставленной цели необходимо выполнить следующие задачи:

\begin{itemize}[label={---}]
	\item описать расстояния Левенштейна и Дамерау --- Левенштейна;
	\item реализовать алгоритмы нахождения расстояний Левенштейна и Дамерау --- Левенштейна;
	
	\item провести тестирование по методу черного ящика для реализаций алгоритмов нахождения расстояний Левенштейна и Дамерау --- Левенштейна;
	
	\item провести сравнительный анализ по времени рекурсивной и матричной реализации алгоритма нахождения расстояния Левенштейна;
	
	\item провести сравнительный анализ по времени матричной и с кешем реализации алгоритма нахождения расстояния Левенштейна;
	
	\item провести сравнительный анализ по времени алгоритмов нахождения расстояния Левенштейна и Дамерау --- Левенштейна;
	
	\item описать и обосновать полученные результаты в отчете о выполненной лабораторной работе, выполненного как расчётно-пояснительная записка к работе.
\end{itemize}