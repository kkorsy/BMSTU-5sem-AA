\chapter*{Заключение}
\addcontentsline{toc}{chapter}{Заключение}

В результате исследования было определено, что время работы реализаций алгоритмов Левенштейна и Дамерау --- Левенштейна растет в геометрической прогрессии при увеличении длин строк. Но лучшие показатели по времени у матричной реализации алгоритма Левенштейна и его оптимизированной по памяти реализации. Также лучшие показатели по памяти у оптимизированной реализации нахождения расстояния Левенштейна. Худшие показатели по времени работы и по памяти у рекурсивной реализации вычисления расстояния Дамерау --- Левенштейна.

Цель, которая была поставлена в начале лабораторной работы была достигнута: исследованы алгоритмы нахождения расстояний Левенштейна и Дамерау --- Левенштейна. В ходе выполнения были решены все задачи:

\begin{itemize}[label={---}]
	\item описаны расстояния Левенштейна и Дамерау --- Левенштейна;
	\item реализованы алгоритмы нахождения расстояний Левенштейна и Дамерау --- Левенштейна;
	
	\item проведено тестирование по методу черного ящика для реализаций алгоритмов нахождения расстояний Левенштейна и Дамерау --- Левенштейна;
	
	\item проведен сравнительный анализ по времени рекурсивной и матричной реализации алгоритма нахождения расстояния Левенштейна;
	
	\item проведен сравнительный анализ по времени матричной и с кешем реализации алгоритма нахождения расстояния Левенштейна;
	
	\item проведен сравнительный анализ по времени алгоритмов нахождения расстояния Левенштейна и Дамерау --- Левенштейна;
	
	\item подготовлен отчет о лабораторной работе.
\end{itemize}