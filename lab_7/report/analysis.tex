\chapter{Аналитическая часть}
В данном разделе будут рассмотрены алгоритмы поиска в двоичном и АВЛ-дереве.

\section{Двоичное дерево поиска}

Дерево двоичного поиска — это структура данных, которая позволяет быстро работать с отсортированном списком чисел \cite{ddp}. Характерные особенности двоичного дерева поиска:

\begin{itemize}[label=---]
	\item Все узлы левого поддерева меньше корневого узла;
	\item Все узлы правого поддерева больше корневого узла;
	\item Оба поддерева каждого узла тоже являются деревьями двоичного поиска, то есть также обладают первыми двумя свойствами. 
\end{itemize}

\textbf{Поиск элемента.}

Если значение меньше корня, то оно находится не в правом поддереве. Поиск выполняется в левом поддереве. А если значение больше корня, то значения нет в левом поддереве. Тогда поиск выполняется в правом поддереве.

\textit{Лучший случай:} искомый элемент расположен в корне дерева. Сложность: $O(1)$.

\textit{Худший случай:} искомый элемент расположен в листе дерева или его нет в дереве. Сложность: $O(n)$, где $n$ --- высота дерева.

\section{АВЛ-дерево}

АВЛ-дерево отличается от обычного бинарного дерева поиска несколькими особенностями \cite{avl}:

\begin{itemize}[label=---]
	\item оно сбалансировано по высоте. Поддеревья, которые образованы левым и правым потомками каждого из узлов, должны различаться длиной не более чем на один уровень;
	\item из первой особенности вытекает еще одна --- общая длина дерева и, соответственно, скорость операций с ним зависят от числа узлов логарифмически и гарантированно;
	\item вероятность получить сильно несбалансированное АВЛ-дерево крайне мала, а риск, что оно выродится, практически отсутствует.
\end{itemize}

\textbf{Поиск элемента} аналогичен поиску в двоичном дереве.

\textit{Лучший случай:} искомый элемент расположен в корне дерева. Сложность: $O(1)$.

\textit{Худший случай:} искомый элемент расположен в листе дерева или его нет в дереве. Сложность: $O(n)$, где $n$ --- высота дерева.

Отличие в том, что высота АВЛ-дерева будет меньше или равна высоте двоичного дерева, а значит поиск будет выполняться быстрее. 

\section{Вывод}

В данном разделе были теоретически разобраны алгоритмы поиска в двоичном и АВЛ-дереве

К разрабатываемой программе предъявляются следующие требования:
\begin{enumerate}
	\item Программа должна предоставлять функциональность поиска элемента в дереве;
	
	\item Реализуемое ПО будет работать в двух режимах --- пользовательском, в котором можно выбрать алгоритм и вывести для него результат, а также экспериментальном режиме, в котором можно произвести сравнение реализаций алгоритмов по количеству сравнений на различных входных данных;
	
	\item В первом режиме в качестве входных данных в программу будет подаваться двоичное дерево, также реализовано меню для вызова алгоритмов и замеров времени. Программа должна корректно обрабатывать случай поиска несуществующего элемента дерева;
	
	\item Во втором режиме будет происходить измерение количества сравнений, будут построены зависимости количества элементов дерева от количества сравнений при поиске. Деревья будут сгенерированы автоматически для заданного количества элементов.
\end{enumerate}