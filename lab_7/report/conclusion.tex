\chapter*{ЗАКЛЮЧЕНИЕ}
\addcontentsline{toc}{chapter}{ЗАКЛЮЧЕНИЕ}

В результате было получено, что поиск в АВЛ-дереве требует меньше сравнений, чем в двоичном дереве в связи с тем, что высота АВЛ-дерева меньше высоты двоичного дерева. При количестве элементов в дереве 1024 и более для поиска элемента в двоичном дереве требуется в 2 раза больше сравнений, чем для поиска в АВЛ-дереве.

Цель, которая была поставлена в начале лабораторной работы была достигнута: изучены алгоритмов поиска. В ходе выполнения были решены все задачи:

\begin{itemize}[label=---]
	\item описаны алгоритмы поиска в двоичном и АВЛ-дереве;
	\item реализованы указанные алгоритмы;
	
	\item проведено тестирование по методу черного ящика для реализаций указанных алгоритмов;
	
	\item проведен сравнительный анализ зависимости количества элементов в дереве от количества сравнений;
	
	\item описаны и обоснованы полученные результаты в отчете о выполненной лабораторной работе, выполненного как расчетно-пояснительная записка к работе.
\end{itemize}