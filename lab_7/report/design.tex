\chapter{Конструкторская часть}

В этом разделе будут представлены псевдокоды алгоритмов поиска в двоичном и АВЛ-дереве.

\section{Разработка алгоритмов}

Поиск в двоичном и АВЛ-дереве выполняется по одному алгоритму. В листинге~\ref{lst:search} представлен псевдокод алгоритмов поиска в двоичном и АВЛ-дереве.

\begin{center}
	\captionsetup{justification=raggedright,singlelinecheck=off}
	\begin{lstlisting}[label=lst:search,caption=Псевдокод алгоритма поиска в двоичном (АВЛ) дереве]
Функция поиска (дерево, значение):
	если дерево - пусто:
		возврат "Элемент не найден"
	если значение равно значению корневого узла дерева:
		возврат "Элемент найден"
	если значение меньше значения корневого узла:
		вызвать функцию поиска (левое поддерево, значение)
	иначе:
		вызвать функцию поиска (правое поддерево, значение)
	\end{lstlisting}
\end{center}

\section{Классы эквивалентности тестирования}

Для тестирования выделены следующие классы эквивалентности:
\begin{enumerate}
	\item искомый элемент расположен в корне дерева;
	\item искомый элемент расположен в листе дерева;
	\item двоичное дерево --- вырожденное;
	\item искомого элемента нет в дереве;
	\item дерево состоит из одного узла;
	\item дерево пустое.
\end{enumerate}

\section{Вывод}

В данном разделе были представлены псевдокоды алгоритмов, рассматриваемых в лабораторной работе.