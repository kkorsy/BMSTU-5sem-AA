\chapter{Технологическая часть}

В данном разделе будут рассмотрены средства реализации, а также представлены листинги реализаций алгоритмов поиска в двоичном и АВЛ-дереве.

\section{Средства реализации}

В данной работе для реализации был выбран язык программирования \textit{С++} \cite{cpp}. 

\section{Описание используемых типов данных}

При реализации будут использованы следующие типы и структуры данных:
\begin{itemize}[label=---]
	\item \textit{Node} --- структура для узла двоичного дерева, содержащая 3 поля:
	\begin{itemize}
		\item \textit{int} --- поле для данных;
		\item \textit{Node*} --- указатель на левое поддерево;
		\item \textit{Node*} --- указатель на правое поддерево;
	\end{itemize}
	\item \textit{NodeAvl} --- структура для узла АВЛ-дерева, содержащая 4 поля:
	\begin{itemize}
		\item \textit{int} --- поле для данных;
		\item \textit{NodeAvl*} --- указатель на левое поддерево;
		\item \textit{NodeAvl*} --- указатель на правое поддерево;
		\item \textit{int} --- поле, содержащее высоту дерева;
	\end{itemize}
\end{itemize}

\newpage
\section{Сведения о модулях программы}

Программа состоит из трех модулей:
\begin{itemize}[label={---}]
	\item \textit{main.cpp} --- файл, содержащий код для выбора алгоритма, а также замеров количества сравнений;
	\item \textit{avl\_tree.cpp} --- файл, обеспечивающий взаимодействие с АВЛ-деревом;
	\item \textit{tree.cpp} --- файл, обеспечивающий взаимодействие с двоичным деревом;
\end{itemize}

\section{Реализация алгоритмов}

В листингах~\ref{lst:avl}~--~\ref{lst:ddp2} представлены реализации алгоритмов поиска в двоичном и АВЛ-дереве.

\lstinputlisting[label=lst:avl, caption={Реализация алгоритма поиска в АВЛ-дереве}, firstline=178, lastline=192]{../avl\_tree.cpp}

\lstinputlisting[label=lst:ddp1, caption={Реализация алгоритма поиска в двоичном дереве (начало)}, firstline=30, lastline=35]{../tree.cpp}

\lstinputlisting[label=lst:ddp2, caption={Реализация алгоритма поиска в двоичном дереве (окончание)}, firstline=35, lastline=44]{../tree.cpp}

\section{Функциональные тесты}

В таблице \ref{tbl:functional_test} приведены тесты для функций, реализующих алгоритмы поиска в двоичном и АВЛ-дереве. В качестве результата ожидается сообщение (найден ли элемент). Тесты \textit{для всех алгоритмов} пройдены успешно. 

\begin{center}
	\captionsetup{justification=raggedright,singlelinecheck=off}
	\begin{longtable}[c]{|c|c|c|c|c|}
		\caption{Функциональные тесты\label{tbl:functional_test}} \\ \hline
		\begin{tabular}{c}
			Значения,\\содержащиеся\\в узлах дерева
		\end{tabular} &
		\begin{tabular}{c}
			Искомый\\элемент
		\end{tabular}
		 & \begin{tabular}{c}
		 	Ожидаемый\\результат
		 \end{tabular} & \begin{tabular}{c}
		 Результат\\поиска в\\АВЛ-дереве
		 \end{tabular} & \begin{tabular}{c}
		 Результат\\поиска в\\двоичном\\дереве
		 \end{tabular} \\ \hline
		\{0, 1, 2, 3\} & 0 & \begin{tabular}{c}
			<<Элемент\\найден>>
		\end{tabular} & \begin{tabular}{c}
		<<Элемент\\найден>>
		\end{tabular} & \begin{tabular}{c}
		<<Элемент\\найден>>
		\end{tabular} \\ \hline
		
		\{0, 1, 2, 3\} & 3 & \begin{tabular}{c}
			<<Элемент\\найден>>
		\end{tabular} & \begin{tabular}{c}
			<<Элемент\\найден>>
		\end{tabular} & \begin{tabular}{c}
			<<Элемент\\найден>>
		\end{tabular} \\ \hline
		
		\{0\} & 0 & \begin{tabular}{c}
			<<Элемент\\найден>>
		\end{tabular} & \begin{tabular}{c}
			<<Элемент\\найден>>
		\end{tabular} & \begin{tabular}{c}
			<<Элемент\\найден>>
		\end{tabular} \\ \hline
		
		\{0\} & 1 & \begin{tabular}{c}
			<<Элемент\\не найден>>
		\end{tabular} & \begin{tabular}{c}
			<<Элемент\\не найден>>
		\end{tabular} & \begin{tabular}{c}
			<<Элемент\\не найден>>
		\end{tabular} \\ \hline
		
		\{\} & 1 & \begin{tabular}{c}
			<<Элемент\\не найден>>
		\end{tabular} & \begin{tabular}{c}
			<<Элемент\\не найден>>
		\end{tabular} & \begin{tabular}{c}
			<<Элемент\\не найден>>
		\end{tabular} \\ \hline
		
		\{0, 1, 2, 3\} & 5 & \begin{tabular}{c}
			<<Элемент\\не найден>>
		\end{tabular} & \begin{tabular}{c}
			<<Элемент\\не найден>>
		\end{tabular} & \begin{tabular}{c}
			<<Элемент\\не найден>>
		\end{tabular} \\ \hline
		
	\end{longtable}
\end{center}

\section{Вывод}

Были представлены реализации алгоритмов поиска в двоичном и АВЛ-дереве, которые были описаны в предыдущем разделе. Также в данном разделе была приведена информация о выбранных средствах для разработки алгоритмов.