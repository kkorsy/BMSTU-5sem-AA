\chapter{Аналитическая часть}
В данном разделе будут рассмотрены блочная сортировка, плавная сортировка и сортировка слиянием.

\section{Блочная сортировка}

Алгоритм блочной (корзинной) сортировки~\cite{bucket-sort} разделяет элементы массива входных данных на некоторое количество блоков --- $k$, количество блоков зависит от количества исходного множества данных. Далее каждый из таких блоков сортируется либо другой сортировкой, либо рекурсивно тем же методом разбиения. После сортировок внутри каждых блоков данные записываются в исходный массив в порядке разбиения на блоки.

\section{Плавная сортировка}

При плавной сортировке~\cite{smooth-sort} в массив накапливается куча из данных, которые затем сортируются путем непрерывного удаления максимума из кучи. В отличие от пирамидальной сортировки, здесь используется не двоичная куча, а специальная, полученная с помощью чисел Леонардо. 

Числа Леонардо --- последовательность чисел, задаваемая зависимостью:

\begin{equation}
	L(n) = \begin{cases}
		1, n = 0\\
		1, n = 1\\
		L(n-1) + L(n-2) + 1, n > 1
	\end{cases}
\end{equation}

Куча состоит из последовательности куч, размеры которых равны одному из чисел Леонардо, а корни хранятся в порядке возрастания. Преимущества таких специальных куч перед двоичными состоят в том, что если последовательность отсортирована, её создание и разрушение займёт $O(n)$ времени, что будет быстрее. 

\section{Сортировка слиянием}

При сортировке слиянием~\cite{merge-sort}, массив разделяется пополам до тех пор, пока каждый участок не станет длиной в один элемент. Затем эти участки возвращаются на место (сливаются) в правильном порядке. В итоге работы алгоритма, исходный массив данных преобразуется в отсортированный, путем сравнивания разделенных элементов между собой. Стоит отметить, что в отличие от линейных алгоритмов сортировки, сортировка слиянием будет делить и склеивать массив вне зависимости от того, был он отсортирован изначально или нет. Поэтому, несмотря на то, что в худшем случае алгоритм отработает быстрее, чем линейный, в лучшем случае, производительность алгоритма будет ниже, чем у линейного. Поэтому сортировка слиянием --- не самое лучшее решение, когда необходимо отсортировать частично упорядоченный массив.

\section{Вывод}

В данном разделе были теоретически разобраны алгоритмы блочной сортировки, плавной сортировки и сортировки слиянием.

К разрабатываемой программе предъявляются следующие требования.

\begin{enumerate}
	\item Программа должна предоставлять функциональность сортировки массива алгоритмами блочной сортировки, плавной сортировки, сортировки слиянием.
	
	\item Реализуемое ПО будет работать в двух режимах --- пользовательском, в котором можно выбрать алгоритм и вывести для него отсортированный массив, а также экспериментальном режиме, в котором можно произвести сравнение реализаций алгоритмов по времени работы на различных входных данных.
	
	\item В первом режиме в качестве входных данных в программу будет подаваться массив, также реализовано меню для вызова алгоритмов и замеров времени. Программа должна корректно обрабатывать случай ввода массива нулевой длины.
	
	\item Во втором режиме будет происходить измерение процессорного времени работы программы, будут построены зависимости времени работы от размера массива. Входной массив будет сгенерирован автоматически для заданного размера массива.
\end{enumerate}