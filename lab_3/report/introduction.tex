\chapter*{Введение}
\addcontentsline{toc}{chapter}{Введение}

В программировании алгоритмы сортировок занимают отдельное важное
место. Их существует огромное количество, и все они обладают различными
свойствами, такими как трудоемкость и объем требуемой памяти.
Упорядочивание элементов в массиве необходимо для решения множества
задач в разных сферах, связанных с математикой, физикой, компьютерной
графикой и т. д.
В данном отчете будут рассмотрены следующие методы сортировки:

\begin{itemize}[label=---]
	\item блочная сортировка;
	\item плавная сортировка;
	\item сортировка слиянием.
\end{itemize}

\textbf{Целью данной работы} является описание, реализация и исследование алгоритмов сортировок --- блочной, плавной и слиянием. Для достижения поставленной цели необходимо выполнить следующие задачи:

\begin{itemize}[label=---]
	\item описать алгоритмы сортировок --- блочной, плавной, слиянием;
	
	\item реализовать алгоритмы указанных сортировок;
	
	\item провести тестирование по методу черного ящика для реализаций указанных алгоритмов сортировок;
	
	\item провести сравнительный анализ по времени и по памяти реализаций указанных алгоритмов сортировок;
	
	\item описать и обосновать полученные результаты в отчете о выполненной лабораторной работе, выполненного как расчетно-пояснительная записка к работе.
\end{itemize}