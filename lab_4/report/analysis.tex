\chapter{Аналитическая часть}
В данном разделе будут рассмотрены последовательный и параллельный алгоритмы Кнута~---~Морриса~---~Пратта. 

\section[Последовательный алгоритм Кнута~---~Морриса~---~Пратта]{Последовательный алгоритм \\Кнута~---~Морриса~---~Пратта}

Алгоритм Кнута~---~Морриса~---~Пратта используется для поиска специальных подстрок с повторами префиксов \cite{kmp}. Его основная идея --- построение автомата для определения величин смещения. При смещении подстроки после $n$ успешных сравнений и 1 неуспешного считается, что одно сравнение префикса в активе и его не нужно проверять.

Рассмотрим пример. Пусть искомая подстрока --- $ababcb$. В построенном автомате состояния маркируются проверяемым в нем символом, дугу --- успехом ($s$) или неудачей ($f$). Полученный автомат представлен на рисунке~\ref{img:auto}.

\inputPdf{auto}{Автомат для определения величин смещения}

В результате будет получен массив сдвигов, который рассчитывается один раз и используется повторно. Для примера выше таким массивом будет:~$[1, 1, 2, 2, 2, 5]$

\section[Параллельный алгоритм Кнута~---~Морриса~---~Пратта]{Параллельный алгоритм \\Кнута~---~Морриса~---~Пратта}

Идея параллельной версии алгоритма Кнута~---~Морриса~---~Пратта --- разбить строку, в которой выполняется поиск, на $n$ равных  частей, где $n$ --- количество потоков. Тогда каждый поток будет искать подстроку в своей части строки.

Однако возможна ситуация, когда исходная строка разделилась таким образом, что части искомой подстроки оказались в разных частях исходной строки, и тогда вхождение не будет найдено. Значит, для корректного решения необходимо делить исходную строку на $n$ частей и к каждой части в начале и в конце добавлять $k$ символов, где $k$ --- длина искомой подстроки. В таком случае возможно повторное нахождение подстроки, поэтому при добавлении найденного индекса в результат необходима проверка на уникальность добавляемого значения.

Поскольку $n$ потоков выполняют запись в один результирующий массив найденных индексов, то необходимо средство синхронизации --- мьютекс. Мьютексы представляют собой объекты ядра, используемые для синхронизации, регулирующие доступ к единственному ресурсу \cite{mutex}.

\section{Вывод}

В данном разделе были теоретически разобраны последовательный и параллельный алгоритмы Кнута~---~Морриса~---~Пратта. 

К разрабатываемой программе предъявляются следующие требования.
\begin{enumerate}
	\item Программа должна предоставлять функциональность поиска подстроки в строке;
	
	\item Реализуемое ПО будет работать в двух режимах --- пользовательском, в котором можно выбрать алгоритм и вывести для него результат, а также экспериментальном режиме, в котором можно произвести сравнение реализаций алгоритмов по времени работы;
	
	\item В первом режиме в качестве входных данных в программу будет подаваться файл, содержащий строку, в которой проводится поиск, также реализовано меню для вызова алгоритмов и замеров времени. Программа должна корректно обрабатывать случай ввода подстроки, которой нет в исходной строке;
	
	\item Во втором режиме будет происходить измерение процессорного времени работы программы, будут построены зависимости времени работы от количества рабочих потоков.
\end{enumerate}