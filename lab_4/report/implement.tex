\chapter{Технологическая часть}

В данном разделе будут рассмотрены средства реализации, а также представлены листинги реализаций последовательного и параллельного алгоритмов Кнута~---~Морриса~---~Пратта.

\section{Средства реализации}

В данной работе для реализации был выбран язык программирования
$C++$ \cite{cpp}. В текущей лабораторной работе требуется замерить процессорное
время для выполняемой программы, а также реализовать принципы многопоточного алгоритма. Все эти инструменты присутствуют в выбранном языке
программирования.
Время замерено с помощью функции $std::chrono::system\_clock::now()$
из библиотеки \textit{chrono} \cite{chrono}.

\section{Описание используемых типов данных}

При реализации будут использованы следующие типы и структуры данных:
\begin{itemize}[label=---]
	\item \textit{std::set} для результирующего набора индексов;
	\item \textit{std::string} для исходной строки, подстроки;
	\item \textit{std::thread} для потока;
	\item \textit{int} для количества потоков. 
\end{itemize}

\section{Сведения о модулях программы}

Программа состоит из одного модуля:
\begin{itemize}[label={---}]
	\item \textit{main.cpp} --- файл, содержащий последовательную и параллельную реализации алгоритмов Кнута~---~Морриса~---~Пратта.
\end{itemize}

\section{Реализация алгоритмов}

В листингах~\ref{lst:posl}~---~\ref{lst:par} представлены последовательная и параллельная реализации алгоритмов Кнута~---~Морриса~---~Пратта.

\lstinputlisting[label=lst:posl, caption={Реализация последовательного алгоритма Кнута~---~Морриса~---~Пратта (начало)}, firstline=104, lastline=134]{../main.cpp}

\lstinputlisting[label=lst:posl, caption={Реализация последовательного алгоритма Кнута~---~Морриса~---~Пратта (окончание)}, firstline=135, lastline=152]{../main.cpp}

\lstinputlisting[label=lst:par, caption={Реализация параллельного алгоритма Кнута~---~Морриса~---~Пратта (начало)}, firstline=72, lastline=87]{../main.cpp}

\lstinputlisting[label=lst:par, caption={Реализация параллельного алгоритма Кнута~---~Морриса~---~Пратта (окончание)}, firstline=87, lastline=103]{../main.cpp}

\section{Функциональные тесты}

В таблице \ref{tbl:functional_test} приведены тесты для функций, реализующих последовательный и параллельный алгоритмы Кнута~---~Морриса~---~Пратта. Тесты \textit{для всех алгоритмов} пройдены успешно. 

\begin{center}
	\captionsetup{justification=raggedright,singlelinecheck=off}
	\begin{longtable}[c]{|c|c|c|c|}
		\caption{Функциональные тесты\label{tbl:functional_test}} \\ \hline
		\begin{tabular}{c}
			Исходная\\строка
		\end{tabular} & \begin{tabular}{c}
		Искомая\\подстрока
		\end{tabular} & \begin{tabular}{c}
		Последовательная\\реализация
		\end{tabular} & \begin{tabular}{c}
		Параллельная\\реализация
		\end{tabular} \\ \hline
		abcd & "пустая строка" & 0 & 0 \\ \hline 
		abab & cd & \begin{tabular}{c}
			<<подстрока\\не найдена>>
		\end{tabular}& \begin{tabular}{c}
		<<подстрока\\не найдена>>
		\end{tabular}\\ \hline
		abababcb & ababcb & 2 & 2 \\ \hline
		abcabd & abc & 0 & 0 \\ \hline
		ababcacab & ab & 0, 2, 7& 0, 2, 7 \\ \hline
	\end{longtable}
\end{center}

\section{Вывод}

Были представлены последовательная и параллельная реализации алгоритмов Кнута~---~Морриса~---~Пратта, которые были описаны в предыдущем разделе. Также в данном разделе была приведена информация о выбранных средствах для разработки алгоритмов.