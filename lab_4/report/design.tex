\chapter{Конструкторская часть}

В этом разделе будут представлены схемы последовательного и параллельного алгоритмов Кнута~---~Морриса~---~Пратта.

\section{Разработка алгоритмов}

На рисунках~\ref{img:kmp1}~---~\ref{img:kmpPar1} представлены схемы схемы последовательного и параллельного алгоритмов Кнута~---~Морриса~---~Пратта. В последовательном алгоритме используются мьютексы поскольку впоследствии данная функция используется для параллельного алгоритма.

\inputPdf{kmp1}{Схема последовательного алгоритма Кнута~---~Морриса~---~Пратта (часть 1)}

\inputPdf{kmp2}{Схема последовательного алгоритма Кнута~---~Морриса~---~Пратта (часть 2)}

\inputPdf{kmpPar1}{Схема параллельного алгоритма Кнута~---~Морриса~---~Пратта}

\newpage
\section{Классы эквивалентности тестирования}

Для тестирования выделены следующие классы эквивалентности:
\begin{enumerate}
	\item ввод подстроки, которой нет в исходной строке;
	\item ввод пустой подстроки;
	\item ввод подстроки, содержащей повторяющуюся часть;
	\item ввод подстроки, состоящей из одного символа;
	\item ввод произвольной подстроки.
\end{enumerate}

\section{Вывод}

В данном разделе были представлены схемы алгоритмов, рассматриваемых в лабораторной работе.