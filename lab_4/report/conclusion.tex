\chapter*{Заключение}
\addcontentsline{toc}{chapter}{Заключение}

В результате было получено, что 32 потока является оптимальным количеством для использованной длины исходной строки. Также было установлено, что в параллельной реализации с фиксированным количеством потоков нет зависимости от длины исходной строки, а в последовательной есть --- линейная. При длине исходной строки выше 800 000 параллельная реализация с 32 потоками будет выполняться быстрее последовательной.

Цель, которая была поставлена в начале лабораторной работы была достигнута: исследованы параллельные вычисления на основе алгоритма Кнута~---~Морриса~---~Пратта. В ходе выполнения лабораторной работы были решены все задачи:
\begin{itemize}[label=---]
	\item описаны последовательный и параллельный алгоритмы Кнута~---~Морриса~---~Пратта;
	\item реализованы указанные алгоритмы;
	\item проведено тестирование по методу черного ящика для реализаций указанных алгоритмов;
	\item выполнен сравнительный анализ зависимостей времени решения от размерности входа для реализации параллельного алгоритма;	
	\item описаны и обоснованы полученные результаты в отчете о выполненной лабораторной работе, выполненного как расчетно-пояснительная записка к работе.
\end{itemize}