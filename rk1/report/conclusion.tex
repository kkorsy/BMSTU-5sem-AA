\chapter*{Заключение}
\addcontentsline{toc}{chapter}{Заключение}

В результате было получено, что в лучшем случае реализация алгоритма Кнута~---~Морриса~---~Пратта работает за $O(1)$, а в худшем за $O(n + m)$, где $n$ --- длина искомой подстроки, $m$ --- длина исходной строки.

Цель, которая была поставлена в начале лабораторной работы была достигнута: алгоритм Кнута~---~Морриса~---~Пратта был применен для поиска подстрок в бинарных строках. Для достижения поставленной цели были выполнены все задачи:
\begin{itemize}[label=---]
	\item описан алгоритм Кнута~---~Морриса~---~Пратта;
	\item реализован указанный алгоритм;
	\item рассчитана трудоемкость в лучшем и худшем случаях;
	\item проведено тестирование по методу черного ящика для реализации указанного алгоритма;
	\item описаны и обоснованы полученные результаты в отчете о выполненной лабораторной работе, выполненного как расчетно-пояснительная записка к работе.
\end{itemize}