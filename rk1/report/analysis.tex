\chapter{Аналитическая часть}
В данном разделе будет рассмотрен алгоритм Кнута~---~Морриса~---~Пратта. 

\section[Алгоритм Кнута~---~Морриса~---~Пратта]{Алгоритм Кнута~---~Морриса~---~Пратта}

Алгоритм Кнута~---~Морриса~---~Пратта используется для поиска специальных подстрок с повторами префиксов \cite{kmp}. Его основная идея --- построение автомата для определения величин смещения. При смещении подстроки после $n$ успешных сравнений и 1 неуспешного считается, что одно сравнение префикса в активе и его не нужно проверять.

Рассмотрим пример. Пусть искомая подстрока --- $ababcb$. В построенном автомате состояния маркируются проверяемым в нем символом, дугу --- успехом ($s$) или неудачей ($f$). Полученный автомат представлен на рисунке~\ref{img:auto}.

\inputPdf{auto}{Автомат для определения величин смещения}

В результате будет получен массив сдвигов, который рассчитывается один раз и используется повторно. Для примера выше таким массивом будет:~$[1, 1, 2, 2, 2, 5]$

\section{Вывод}

В данном разделе был теоретически разобран алгоритм Кнута~---~Морриса~---~Пратта. 

К разрабатываемой программе предъявляются следующие требования:
\begin{enumerate}
	\item Программа должна предоставлять функциональность поиска бинарной подстроки в строке;
	
	\item Реализуемое ПО будет работать в одном режиме --- пользовательском, в котором можно вывести результат работы алгоритма;
	
	\item В качестве входных данных в программу будет подаваться файл, содержащий бинарную строку, в которой проводится поиск, также реализовано меню для вызова алгоритма и замеров времени. Программа должна корректно обрабатывать случай ввода подстроки, которой нет в исходной строке;
\end{enumerate}