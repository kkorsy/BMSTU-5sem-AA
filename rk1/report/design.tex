\chapter{Конструкторская часть}

В этом разделе будет представлена схема алгоритма Кнута~---~Морриса~---~Пратта, а также рассчитана трудоемкость алгоритма в лучшем и худшем случаях.

\section{Разработка алгоритмов}

На рисунках~\ref{img:kmp1}~--~\ref{img:kmp2} представлена схема алгоритма Кнута~---~Морриса~---~Пратта. 

\inputPdf{kmp1}{Схема алгоритма Кнута~---~Морриса~---~Пратта (часть 1)}

\inputPdf{kmp2}{Схема алгоритма Кнута~---~Морриса~---~Пратта (часть 2)}

\newpage
\section{Классы эквивалентности тестирования}

Для тестирования выделены следующие классы эквивалентности:
\begin{enumerate}
	\item ввод подстроки, которой нет в исходной строке;
	\item ввод пустой подстроки;
	\item ввод подстроки, содержащей повторяющуюся часть;
	\item ввод подстроки, состоящей из одного символа;
	\item ввод произвольной подстроки.
\end{enumerate}

\section{Модель вычислений}

\begin{enumerate}
	\item Трудоемкость базовых операций.
	
	Следующие операторы имеют трудоемкость 1:
	$$+, -, +=, -=, =, ==, !=, >=, <=, >, <,$$
	$$>>, <<, [], \&, |, \&\&, ||, ++, --$$
	
	Следующие операторы имеют трудоемкость 2:
	$$*, /, \%, *=, /=$$
	
	\item Условный оператор.
	
	Для конструкций вида:
	\begin{lstlisting}
		if (условие)
		{
			Блок 1;
		}
		else
		{
			Блок 2;
		}
	\end{lstlisting}
	Пусть трудоемкость блока 1 --- $f_1$, блока 2 --- $f_2$. Пусть также трудоемкость условного перехода --- 0.
	
	Тогда трудоемкость условного оператора:
	\begin{equation}
		f_{if} = f_\textup{вычисления условия} + 
		\left[ \begin{gathered}
			\min(f_1, f_2), \textup{лучший случай}\\
			\max(f_1, f_2), \textup{худший случай}
		\end{gathered}
		\right.
	\end{equation}
	
	\item Трудоемкость циклов.
	
	Трудоемкость циклов вычисляется по следующей формуле:
	
	\begin{equation}
		\begin{aligned}
			f_\textup{цикла} =& f_\textup{инициализации} + f_\textup{сравнения} + \\
			& + M_\textup{шагов} \cdot (f_\textup{тела} + f_\textup{инкремента} + f_\textup{сравнения})
		\end{aligned}
	\end{equation}
	
\end{enumerate}

\section{Трудоемкость алгоритма}

Рассчитаем трудоемкость алгоритма Кнута~---~Морриса~---~Пратта в лучшем и худшем случаях.

В таблице \ref{tbl:kmp} представлена построчная оценка трудоемкости.
\begin{table}[H]
	\begin{center}
		\begin{threeparttable}
			\captionsetup{}
			\caption{\label{tbl:kmp}Построчная оценка трудоемкости}
			\begin{tabular}{|l|c|}
				\hline
				Строка кода & Вес \\ \hline
				int m = text.length(); & 2\\ \hline
				int n = pattern.length(); & 2\\ \hline
				if (n == 0) & 1\\ \hline
				~~result.insert(0); & 1\\ \hline
				~~return; & 0\\ \hline
				
				if (m < n) & 1\\ \hline
				~~return; & 0\\ \hline
				
				int next[n + 1];& 1\\ \hline
				for (int i = 0; i < n + 1; i++)& 4\\ \hline
				~~next[i] = 0;& 2\\ \hline
				
				for (int i = 1; i < n; i++)& 3\\ \hline
				~~int j = next[i];& 2\\ \hline
				~~while (j > 0 \&\& pattern[j] != pattern[i])& 5\\ \hline
				~~~~j = next[j];& 2\\ \hline
				~~if (j > 0 || pattern[j] == pattern[i])& 5\\ \hline
				~~~~next[i + 1] = j + 1;& 4\\ \hline
				
				for (int i = 0, j = 0; i < m; i++)& 4\\ \hline
				~~if (text[i] == pattern[j])& 3\\ \hline
				~~~~if (++j == n)& 2\\ \hline
				~~~~~~result.insert(start\_pos + i - j + 1);& 4\\ \hline
				~~else if (j > 0)& 1\\ \hline
				~~~~j = next[j];& 2\\ \hline
				~~~~i- -;& 1\\ \hline
			\end{tabular}
		\end{threeparttable}
	\end{center}
\end{table}

\textbf{Лучший случай:} длина искомой подстроки равна 0. Tрудоемкость:
\begin{equation}
	f = O(6) = O(1)
\end{equation}

\textbf{Худший случай:} в исходной строке нет искомой подстроки. Tрудоемкость:
\begin{equation}
	f = O(2m + 2n - 3) = O(m + n)
\end{equation}


\section{Вывод}

В данном разделе была представлена схема алгоритма Кнута~---~Морриса~---~Пратта и рассчитана трудоемкость в лучшем и худшем случаях.