\chapter{Технологическая часть}

В данном разделе будут рассмотрены средства реализации, а также представлены листинги реализации алгоритма Кнута~---~Морриса~---~Пратта.

\section{Средства реализации}

В данной работе для реализации был выбран язык программирования $C++$ \cite{cpp}.

\section{Описание используемых типов данных}

При реализации будут использованы следующие типы и структуры данных:
\begin{itemize}[label=---]
	\item \textit{std::set} для результирующего набора индексов;
	\item \textit{char *} для исходной строки;
	\item \textit{std::string} для подстроки.
\end{itemize}

\section{Сведения о модулях программы}

Программа состоит из одного модуля:
\begin{itemize}[label={---}]
	\item \textit{main.cpp} --- файл, содержащий реализацию алгоритма Кнута~---~Морриса~---~Пратта.
\end{itemize}

\section{Реализация алгоритмов}

В листинге~\ref{lst:kmp} представлена реализация алгоритма Кнута~---~Морриса~---~Пратта для бинарных строк.

\lstinputlisting[label=lst:kmp, caption={Реализация алгоритма Кнута~---~Морриса~---~Пратта для бинарных строк}, firstline=62, lastline=106]{../main.cpp}

\section{Функциональные тесты}

В таблице \ref{tbl:functional_test} приведены тесты для функции, реализующей алгоритм Кнута~---~Морриса~---~Пратта. Тесты пройдены успешно. 

\begin{center}
	\captionsetup{justification=raggedright,singlelinecheck=off}
	\begin{longtable}[c]{|c|c|c|}
		\caption{Функциональные тесты\label{tbl:functional_test}} \\ \hline
		\begin{tabular}{c}
			Исходная\\строка
		\end{tabular} & \begin{tabular}{c}
		Искомая\\подстрока
		\end{tabular} & \begin{tabular}{c}
		Реализация алгоритма\\Кнута~---~Морриса~---~Пратта
		\end{tabular}\\ \hline
		abcd & "пустая строка" & 0 \\ \hline 
		abab & cd & \begin{tabular}{c}
			<<подстрока\\не найдена>>
		\end{tabular}\\ \hline
		abababcb & ababcb & 2 \\ \hline
		abcabd & abc & 0 \\ \hline
		ababcacab & ab & 0, 2, 7\\ \hline
	\end{longtable}
\end{center}

\section{Вывод}

Была представлена реализация алгоритма Кнута~---~Морриса~---~Пратта, который был описан в предыдущем разделе. Также в данном разделе была приведена информация о выбранных средствах для разработки алгоритма.